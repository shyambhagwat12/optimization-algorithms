C. Albert Thompson
Home
Papers
Classes
Tools
FAQ
Online Latex Formatter
%%%%%%%%%%%%%%%%%%%%%%%%%%%%%%%%%%%%%%%%%%%%%%%%%%%%%%%%%%%%%%%%%%%%%
% Use the koma-script document style
%\documentclass{scrbook}
%\KOMAoptions{twoside=false} % disable two-side formatting for scrbook
% alternatively, for shorter essay, use the following
\documentclass{scrartcl}
%%%%%%%%%%%%%%%%%%%%%%%%%%%%%%%%%%%%%%%%%%%%%%%%%%%%%%%%%%%%%%%%%%%%%

%%%%%%%%%%%%%%%%%%%%%%%%%%%%%%%%%%%%%%%%%%%%%%%%%%%%%%%%%%%%%%%%%%%%%
% Useful packages
\usepackage{mathtools}
\usepackage{amssymb,bm,bbold}
\usepackage[colorlinks=true]{hyperref}
\usepackage{enumerate}

\usepackage{fullpage}

\usepackage[font=scriptsize,labelfont=bf]{caption}
\usepackage{wrapfig}
\usepackage{tikz}
\usepackage{pgfplots}

\usepackage[]{natbib}
\bibliographystyle{chicago}

%=================================
% pre-defined theorem environments
\usepackage{amsthm}
\newtheorem{theorem}{Theorem}
\newtheorem{lemma}{Lemma}
\newtheorem{proposition}{Proposition}
\newtheorem{corollary}{Corollary}
\newtheorem{definition}{Definition}
\newtheorem*{remark}{Remark}
\newtheorem*{assumption}{Assumption}

\usepackage[framemethod=tikz]{mdframed}
\newmdtheoremenv[
skipabove=\baselineskip,
skipbelow=\baselineskip,
hidealllines=true,
innertopmargin=4pt,
linewidth=4pt,
linecolor=gray!40,
singleextra={
  \draw[line width=3pt,gray!50,line cap=rect] (O|-P) -- +(1cm,0pt);
  \draw[line width=3pt,gray!50,line cap=rect] (O|-P) -- +(0pt,-1cm);
  \draw[line width=3pt,gray!50,line cap=rect] (O-|P) -- +(-1cm,0pt);
  \draw[line width=3pt,gray!50,line cap=rect] (O-|P) -- +(0pt,1cm);
  },
firstextra={
  \draw[line width=3pt,gray!50,line cap=rect] (O|-P) -- +(1cm,0pt);
  \draw[line width=3pt,gray!50,line cap=rect] (O|-P) -- +(0pt,-1cm);
},
secondextra={
  \draw[line width=3pt,gray!50,line cap=rect] (O-|P) -- +(-1cm,0pt);
  \draw[line width=3pt,gray!50,line cap=rect] (O-|P) -- +(0pt,1cm);
}
]{problem}{Problem}

%=================================
% useful commands
\DeclareMathOperator*{\argmin}{arg\,min}
\DeclareMathOperator*{\argmax}{arg\,max}
\DeclareMathOperator*{\supp}{supp}
\DeclareMathOperator*{\minimize}{\mathsf{minimize}}

\def\vec#1{{\ensuremath{\bm{{#1}}}}}
\def\mat#1{\vec{#1}}

%=================================
% convenient notations
\newcommand{\XX}{\mathbb{X}}
\newcommand{\RR}{\mathbb{R}}
\newcommand{\EE}{\mathbb{E}}
\newcommand{\PP}{\mathbb{P}}

\newcommand{\sB}{\mathcal{B}}
\newcommand{\sK}{\mathcal{K}}
\newcommand{\sL}{\mathcal{L}}
\newcommand{\sX}{\mathcal{X}}

\newcommand{\TODO}{\textbf{\textsf{TODO}}}

\usepackage{algpseudocode}
\usepackage{algorithm}

%%%%%%%%%%%%%%%%%%%%%%%%%%%%%%%%%%%%%%%%%%%%%%%%%%%%%%%%%%%%%%%%%%%%%
% Typography, change document font
%\usepackage[libertine,cmintegrals,cmbraces,vvarbb]{newtxmath}
%\usepackage[scaled=0.95]{inconsolata}
\usepackage{lmodern}
\usepackage{charter}
\usepackage[scaled=0.95]{inconsolata}

\title{Intuitions for Convex Optimization}
\author{pluskid}

\begin{document}

Proof. Letting $x=x_{t}$ we have
$$
\begin{aligned}
\left\|x_{t+1}-x^{*}\right\|^{2} &=\left\|x-x^{*}-\gamma \nabla_{x} f\right\|^{2} \\
&=\left\|x-x^{*}\right\|^{2}-2 \gamma\left(\nabla_{x} f\right)^{\top}\left(x-x^{*}\right)+\gamma^{2}\left\|\nabla_{x} f\right\|^{2} \\
&=\left\|x-x^{*}\right\|^{2}-2 \gamma\left[\ell_{x}(x)-\ell_{x}\left(x^{*}\right)\right]+\gamma^{2}\left\|\nabla_{x} f\right\|^{2} \\
& \leq\left\|x-x^{*}\right\|^{2}-2 \gamma\left(f(x)-f\left(x^{*}\right)\right)+\gamma^{2}\left\|\nabla_{x} f\right\|^{2}
\end{aligned}
$$
The proof concludes by observing that the $L$-Lipschitz property of $f$ implies $\left\|\nabla_{x} f\right\| \leq L$.
Let $\Phi(t)=\left\|x^{t}-x^{*}\right\|^{2}$. When $\gamma=\varepsilon / L^{2}$, the lemma implies that whenever $f\left(x_{t}\right)>f\left(x^{*}\right)+\varepsilon$, we have
$$
\Phi(t)-\Phi(t+1)>2 \gamma \varepsilon-\gamma^{2} L^{2}=\varepsilon^{2} / L^{2} .
$$
Since $\Phi(0) \leq D$ and $\Phi(t) \geq 0$ for all $t$, the equation (1) cannot be satisfied for all $0 \leq t \leq L^{2} D^{2} / \varepsilon^{2}$. Hence, if we run gradient descent for $T=L^{2} D^{2} / \varepsilon^{2}$ iterations, it succeeds in finding a point $\tilde{x}$ such that $f(\tilde{x}) \leq f\left(x^{*}\right)+\varepsilon$.
\end{document}

 
%%%%%%%%%%%%%%%%%%%%%%%%%%%%%%%%%%%%%%%%%%%%%%%%%%%%%%%%%%%%%%%%%%%%%
% Use the koma-script document style
%\documentclass{scrbook}
%\KOMAoptions{twoside=false} % disable two-side formatting for scrbook
% alternatively, for shorter essay, use the following
\documentclass{scrartcl}
%%%%%%%%%%%%%%%%%%%%%%%%%%%%%%%%%%%%%%%%%%%%%%%%%%%%%%%%%%%%%%%%%%%%%

%%%%%%%%%%%%%%%%%%%%%%%%%%%%%%%%%%%%%%%%%%%%%%%%%%%%%%%%%%%%%%%%%%%%%
% Useful packages
\usepackage{mathtools}
\usepackage{amssymb,bm,bbold}
\usepackage[colorlinks=true]{hyperref}
\usepackage{enumerate}

\usepackage{fullpage}

\usepackage[font=scriptsize,labelfont=bf]{caption}
\usepackage{wrapfig}
\usepackage{tikz}
\usepackage{pgfplots}

\usepackage[]{natbib}
\bibliographystyle{chicago}

%=================================
% pre-defined theorem environments
\usepackage{amsthm}
\newtheorem{theorem}{Theorem}
\newtheorem{lemma}{Lemma}
\newtheorem{proposition}{Proposition}
\newtheorem{corollary}{Corollary}
\newtheorem{definition}{Definition}
\newtheorem*{remark}{Remark}
\newtheorem*{assumption}{Assumption}

\usepackage[framemethod=tikz]{mdframed}
\newmdtheoremenv[
skipabove=\baselineskip,
skipbelow=\baselineskip,
hidealllines=true,
innertopmargin=4pt,
linewidth=4pt,
linecolor=gray!40,
singleextra={
  \draw[line width=3pt,gray!50,line cap=rect] (O|-P) -- +(1cm,0pt);
  \draw[line width=3pt,gray!50,line cap=rect] (O|-P) -- +(0pt,-1cm);
  \draw[line width=3pt,gray!50,line cap=rect] (O-|P) -- +(-1cm,0pt);
  \draw[line width=3pt,gray!50,line cap=rect] (O-|P) -- +(0pt,1cm);
  },
firstextra={
  \draw[line width=3pt,gray!50,line cap=rect] (O|-P) -- +(1cm,0pt);
  \draw[line width=3pt,gray!50,line cap=rect] (O|-P) -- +(0pt,-1cm);
},
secondextra={
  \draw[line width=3pt,gray!50,line cap=rect] (O-|P) -- +(-1cm,0pt);
  \draw[line width=3pt,gray!50,line cap=rect] (O-|P) -- +(0pt,1cm);
}
]{problem}{Problem}

%=================================
% useful commands
\DeclareMathOperator*{\argmin}{arg\,min}
\DeclareMathOperator*{\argmax}{arg\,max}
\DeclareMathOperator*{\supp}{supp}
\DeclareMathOperator*{\minimize}{\mathsf{minimize}}

\def\vec#1{{\ensuremath{\bm{{#1}}}}}
\def\mat#1{\vec{#1}}

%=================================
% convenient notations
\newcommand{\XX}{\mathbb{X}}
\newcommand{\RR}{\mathbb{R}}
\newcommand{\EE}{\mathbb{E}}
\newcommand{\PP}{\mathbb{P}}

\newcommand{\sB}{\mathcal{B}}
\newcommand{\sK}{\mathcal{K}}
\newcommand{\sL}{\mathcal{L}}
\newcommand{\sX}{\mathcal{X}}

\newcommand{\TODO}{\textbf{\textsf{TODO}}}

\usepackage{algpseudocode}
\usepackage{algorithm}

%%%%%%%%%%%%%%%%%%%%%%%%%%%%%%%%%%%%%%%%%%%%%%%%%%%%%%%%%%%%%%%%%%%%%
% Typography, change document font
%\usepackage[libertine,cmintegrals,cmbraces,vvarbb]{newtxmath}
%\usepackage[scaled=0.95]{inconsolata}
\usepackage{lmodern}
\usepackage{charter}
\usepackage[scaled=0.95]{inconsolata}

\title{Intuitions for Convex Optimization}
\author{pluskid}

\begin{document}

Proof. Letting $x=x_{t}$ we have
$$
\begin{aligned}
	\left\|x_{t+1}-x^{*}\right\|^{2} & =\left\|x-x^{*}-\gamma \nabla_{x} f\right\|^{2}                                                                                    \\
	                                 & =\left\|x-x^{*}\right\|^{2}-2 \gamma\left(\nabla_{x} f\right)^{\top}\left(x-x^{*}\right)+\gamma^{2}\left\|\nabla_{x} f\right\|^{2} \\
	                                 & =\left\|x-x^{*}\right\|^{2}-2 \gamma\left[\ell_{x}(x)-\ell_{x}\left(x^{*}\right)\right]+\gamma^{2}\left\|\nabla_{x} f\right\|^{2}  \\
	                                 & \leq\left\|x-x^{*}\right\|^{2}-2 \gamma\left(f(x)-f\left(x^{*}\right)\right)+\gamma^{2}\left\|\nabla_{x} f\right\|^{2}             
\end{aligned}
$$
The proof concludes by observing that the $L$-Lipschitz property of $f$ implies $\left\|\nabla_{x} f\right\| \leq L$.
Let $\Phi(t)=\left\|x^{t}-x^{*}\right\|^{2}$. When $\gamma=\varepsilon / L^{2}$, the lemma implies that whenever $f\left(x_{t}\right)>f\left(x^{*}\right)+\varepsilon$, we have
$$
\Phi(t)-\Phi(t+1)>2 \gamma \varepsilon-\gamma^{2} L^{2}=\varepsilon^{2} / L^{2} .
$$
Since $\Phi(0) \leq D$ and $\Phi(t) \geq 0$ for all $t$, the equation (1) cannot be satisfied for all $0 \leq t \leq L^{2} D^{2} / \varepsilon^{2}$. Hence, if we run gradient descent for $T=L^{2} D^{2} / \varepsilon^{2}$ iterations, it succeeds in finding a point $\tilde{x}$ such that $f(\tilde{x}) \leq f\left(x^{*}\right)+\varepsilon$.
\end{document}
© C. Albert Thompson 2016 | This page was made possible by CMH and latexindent.pl